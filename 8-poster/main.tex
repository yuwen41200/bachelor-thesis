% based on poster for ASE 2009 by Martin Weiglhofer

\documentclass[final,hyperref={pdfpagelabels=false}]{beamer}
\setbeamertemplate{caption}[numbered]
\mode<presentation>{\usetheme{TUGraz}}

\usepackage{CJKutf8}
\usepackage{amsmath,amsthm,amssymb,latexsym}
\usepackage{graphicx}
\usepackage{multirow}
\usepackage{pifont}
\usepackage{algpseudocode}
\usepackage{colortbl}
\usepackage{booktabs}
\usepackage{array}
\usepackage[orientation=portrait,size=a0,scale=1.4]{beamerposter}

\institute[nctucs]{國立交通大學資訊工程學系}
\title{社群網站上之政治活動分析:以台灣{\fontfamily{cmr}\selectfont\,2016\,}年總統選舉為例}
\author[ywpu]{學生:蒲郁文\hspace*{1in}指導教授:孫春在 教授}
\mail{ywpu@cs.nctu.edu.tw}
\webpage{http://ywpu.me/}

\begin{document}
\begin{CJK}{UTF8}{bsmi}
\newcolumntype{L}[1]{>{\raggedright\arraybackslash}p{#1}}
\newcolumntype{C}[1]{>{\centering\arraybackslash}p{#1}}
\newcolumntype{R}[1]{>{\raggedleft\arraybackslash}p{#1}}
\begin{frame}
\centering

\begin{minipage}{0.37\textwidth}
\begin{alertblock}{\textbf{摘要}}
  本研究嘗試使用關鍵字擷取(keyword extraction)、%
  情感分析(sentiment analysis)、社會網絡分析(social network analysis)等方法,%
  分析 2016 年台灣總統選舉競選期間臉書各大政治性粉絲專頁之公開資料,探討使用者於社群網站上之政治活動的模式。%
  本研究發現此次選舉有 26.63\% 的社群媒體貼文是攻擊型競選,與台灣、中國或中國國民黨有關的議題是人們最關注的焦點,%
  且特定族群會傾向於分享特定來源的資訊。%
  同時,本研究也提供了後續研究者一套有效地自動化分析線上政治活動的方法。%
\end{alertblock}
\end{minipage}
\quad
\begin{minipage}{0.61\textwidth}
\begin{block}{總體分佈}
\begin{center}
\begin{minipage}{0.43\textwidth}
  \begin{figure}[!htbp]
  \centering
  \includegraphics[width=\columnwidth]{quantity_time_graph}
  \caption{貼文數量對時間的關係圖}
  \label{f1}
  \end{figure}
\end{minipage}
\quad
\begin{minipage}{0.43\textwidth}
  \begin{figure}[!htbp]
  \centering
  \includegraphics[width=\columnwidth]{quantity_time_cumulative_graph}
  \caption{累計貼文數量對時間的關係圖}
  \label{f2}
  \end{figure}
\end{minipage}
\end{center}
\end{block}
\end{minipage}

\begin{minipage}{0.30\textwidth}
\begin{block}{總體分佈(續)}
  \begin{table}[!htbp]
  \caption{各粉絲專頁的貼文數量(節錄)}
  \label{t1}
  \begin{tabular}{@{}L{12cm}C{5cm}C{5cm}@{}}
  \toprule
  粉絲專頁名稱 & 貼文數量 & 百分比 \\
  \midrule
  柯建銘 & 485 & 2.56\% \\
  Taiwan Fugue & 361 & 1.90\% \\
  台灣新力量/羅致政\dots & 337 & 1.78\% \\
  宋楚瑜找朋友 & 334 & 1.76\% \\
  反馬英九聯盟 & 330 & 1.74\% \\
  林昶佐 Freddy Lim & 320 & 1.69\% \\
  基進黨(基進側翼) & 319 & 1.68\% \\
  BillyPan 潘建志醫師 & 306 & 1.61\% \\
  宋楚瑜 & 301 & 1.59\% \\
  綠黨 & 297 & 1.57\% \\
  民主進步黨 & 293 & 1.54\% \\
  鄭永金後援會 & 282 & 1.49\% \\
  我是台灣人 (I am\dots & 281 & 1.48\% \\
  江啟臣服務讚 & 262 & 1.38\% \\
  \bottomrule
  \end{tabular}
  \end{table}
\end{block}
\end{minipage}
\quad
\begin{minipage}{0.68\textwidth}
\begin{block}{關鍵字擷取}
\begin{minipage}{0.34\textwidth}
  從表 \ref{t2} 我們可以看出各政黨/侯選人/意見領袖重視的議題。%
  例如,中國國民黨關注朱立倫、經濟、兩岸等事務,王金平關注國會、改革等事務,%
  綠黨關注勞工、環境等議題,而馬躍.比吼則關注原住民的文化和土地等等。%
  透過表 \ref{t3},我們可以發現人們關注的焦點皆圍繞在台灣、中國、國民黨上,唯有在 11 月 1 日至 11 月 8 日間,%
  由於舉行兩岸領導人會面,馬英九、馬習會也躍升為社群網站上的討論焦點。%
  表 \ref{t4} 告訴我們本次選舉人們最關心跟台灣和中國有關的議題,而國民黨也是人們討論的焦點。%
  值得一提的是,此次選舉的另一大黨,民主進步黨,並未出現在前十大關鍵字之中。%
\end{minipage}
\quad
\begin{minipage}{0.30\textwidth}
  \begin{table}[!htbp]
  \caption{依作者劃分(節錄)}
  \label{t2}
  \begin{tabular}{@{}L{8.5cm}ll@{}}
  \toprule
  粉絲專頁名稱 & \#1 & \#2 \\
  \midrule
  朱立倫 & 台灣 & 新 \\
  大安推范雲 & 范雲 & 台灣 \\
  王丹网站 Wa\dots & 中國 & 中 \\
  國民黨青年團 & 青年 & 台灣 \\
  呂欣潔 Jennif\dots & 政治 & 欣潔 \\
  管碧玲 (kuan\dots & 管媽 & 市長 \\
  段宜康 & 朱立倫 & 國民黨 \\
  鄉民實業坊 & 黑子 & 種 \\
  我是台灣人 (\dots & 台灣 & 綠黨 \\
  林智堅 & 新竹市 & 新竹 \\
  BillyPan 潘\dots & 醫師 & 台灣 \\
  蔡錦隆加油讚 & 蔡錦隆 & 錦隆 \\
  馬英九 & 臺灣 & 中 \\
  醫勞盟 & 醫師 & 醫療 \\
  \bottomrule
  \end{tabular}
  \end{table}
\end{minipage}
\quad
\begin{minipage}{0.29\textwidth}
  \begin{table}[!htbp]
  \caption{依時間劃分(節錄)}
  \label{t3}
  \begin{tabular}{@{}L{7.5cm}ll@{}}
  \toprule
  時間區間 & \#1 & \#3 \\
  \midrule
  11/15 $\sim$ 11/22 & 台灣 & 政府 \\
  11/08 $\sim$ 11/15 & 台灣 & 中國 \\
  11/01 $\sim$ 11/08 & 台灣 & 中國 \\
  10/25 $\sim$ 11/01 & 台灣 & 國民黨 \\
  10/17 $\sim$ 10/25 & 台灣 & 新 \\
  \bottomrule
  \end{tabular}
  \end{table}
  \vspace{1em}
  \begin{table}[!htbp]
  \caption{合併所有貼文(節錄)}
  \label{t4}
  \begin{tabular}{@{}lll@{}}
  \toprule
  \#1 & \#3 & \#7 \\
  \midrule
  台灣  & 國民黨 & 中國  \\
  \bottomrule
  \end{tabular}
  \end{table}
\end{minipage}
\end{block}
\end{minipage}

\begin{minipage}{\textwidth}
\begin{block}{情感分析}
\begin{minipage}{0.31\textwidth}
  本情感分類模型的正確率達 86.82\%。%
  本研究發現,此次選舉只有 26.63\% 的社群媒體貼文是攻擊型競選。%
  許多政治人物都不常使用攻擊型競選,然而,具鮮明政治立場,或以抨擊反對陣營為目的的意見領袖,則較常使用攻擊型競選。%
  Taiwan Fugue、反馬英九聯盟、不禮貌鄉民團、藍白拖的逆襲等粉絲專頁攻擊型競選貼文的比例皆超過八成。%
  對此,有一種可能的情形是,政治人物在其官方粉絲專頁上保持正面競選的形象,但在暗中卻支持其他非官方的粉絲專頁攻擊、批評其反對者。%
  使用攻擊型競選的比例並未隨著時間而有太大的變化,唯有在 11 月 1 日至 11 月 8 日間,%
  或許是因為舉行兩岸領導人會面的緣故,攻擊型競選貼文的比例創下高峰,比平時高出了一成左右。%
  另外,使用攻擊型競選的比例則在投票日前一週達到最低點,比平時低了 5\% 左右。%
\end{minipage}
\quad
\begin{minipage}{0.39\textwidth}
  \begin{table}[!htbp]
  \caption{貼文情感分析(節錄)}
  \label{t5}
  \begin{tabular}{@{}L{8.5cm}R{4.5cm}R{4.5cm}R{1.7cm}R{4.5cm}R{5.7cm}@{}}
  \toprule
   \newline粉絲專頁名稱\newline  & %
   \,01/10\newline\hspace*{1.25em}$\vert$\newline01/17 & %
   \,01/03\newline\hspace*{1.25em}$\vert$\newline01/10 & %
   \newline\,$\ldots$\newline  & %
   \,10/17\newline\hspace*{1.25em}$\vert$\newline10/25 & %
   {\bfseries  所有  時間  區間} \\
  \midrule
  朱立倫 & 1/30 & 1/18 & $\ldots$ & 1/5 & 10/210 \\
  大安推范雲 & 2/35 & 0/27 & $\ldots$ & 1/13 & 14/239 \\
  王丹网站 Wa\dots & 3/14 & 6/11 & $\ldots$ & 5/12 & 37/137 \\
  國民黨青年團 & 1/4 & 0/4 & $\ldots$ & 0/0 & 4/51 \\
  $\vdots$ & $\vdots$ & $\vdots$ & $\ddots$ & $\vdots$ & $\vdots$ \\
  宋楚瑜 & 7/57 & 4/51 & $\ldots$ & 1/22 & 32/294 \\
  特急件小周\dots & 19/25 & 13/20 & $\ldots$ & 4/6 & 99/142 \\
  {\bfseries 所有粉絲專頁} & 509/2355 & 446/1868 & $\ldots$ & 336/1275 & 4827/18128 \\
  \bottomrule
  \end{tabular}
  \end{table}
\end{minipage}
\quad
\begin{minipage}{0.25\textwidth}
  \begin{figure}[!htbp]
  \centering
  \includegraphics[width=\columnwidth]{meta}
  \caption{訓練/測試貼文的分佈圖}
  \label{f3}
  \end{figure}
\end{minipage}
\end{block}
\end{minipage}

\begin{minipage}{0.68\textwidth}
\begin{block}{社會互動}
\begin{minipage}{0.50\textwidth}
  本部份探討的是社群網站上的「分享」網絡,分析使用者分享的貼文/連結的來源,用 Gephi 畫出社會網絡圖。%
  本研究從全部貼文中,一共整理出 12,428 筆分享紀錄。
  從圖 \ref{f4} 我們可以發現一些規則,例如,%
  nagee 較常分享蘋果日報及批踢踢的消息;Taiwan Fugue 較常分享自由時報及蘋果日報的消息;%
  白色正義聯盟較常分享聯合新聞網的消息;我是台灣人 (I am Taiwanese) 較常分享綠黨的消息等等。%
  另外,此圖也呈現出各政黨/侯選人/意見領袖之間的關係。%
  例如苗博雅、呂欣潔和范雲皆屬於同一個政黨,因此他們在圖中也屬於同一個社群。%
  同樣地,白色正義聯盟、挺馬英九聯盟和藍白拖的逆襲彼此有相近的政治立場,因此他們在圖中的位置也比較相近。%
  最後,從圖中圓點的大小,我們還可以看出哪些消息來源最常被人們引用。%
  在此次選舉中最常被人們引用的消息來源是 YouTube、自由時報、蘋果日報和聯合新聞網等等。%
  從這裡我們或許也能看出各消息來源在使用者心中的可信度。%
\end{minipage}
\quad
\begin{minipage}{0.43\textwidth}
  \begin{figure}[!htbp]
  \centering
  \includegraphics[width=\columnwidth]{sna_sm}
  \caption{貼文/連結的「分享」網絡圖}
  \label{f4}
  \end{figure}
\end{minipage}
\end{block}
\end{minipage}
\quad
\begin{minipage}{0.30\textwidth}
\begin{block}{重要參考文獻}
  \nocite{*}
  \bibliographystyle{IEEEtran}
  \bibliography{citation}  
\end{block}
\end{minipage}

\end{frame}
\end{CJK}
\end{document}
