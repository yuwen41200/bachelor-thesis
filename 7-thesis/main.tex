%%%%%%%%%%%%%%%%%%%%%%%%%%%%%%%%%%%%%%%%%%%%%%%%%%%%%%%%%%%%%%%%%%%%%%%%%%%%%%%%
%2345678901234567890123456789012345678901234567890123456789012345678901234567890
%        1         2         3         4         5         6         7         8

\documentclass[letterpaper, 10pt, conference]{ieeeconf}   % Comment this line out
                                                          % if you need a4paper
%\documentclass[a4paper, 10pt, conference]{ieeeconf}      % Use this line for a4
                                                          % paper

\IEEEoverridecommandlockouts                              % This command is only
                                                          % needed if you want to
                                                          % use the \thanks command
\overrideIEEEmargins
% See the \addtolength command later in the file to balance the column lengths
% on the last page of the document

% The following packages can be found on http:\\www.ctan.org
%\usepackage{graphics} % for pdf, bitmapped graphics files
%\usepackage{epsfig} % for postscript graphics files
%\usepackage{mathptmx} % assumes new font selection scheme installed
%\usepackage{times} % assumes new font selection scheme installed
%\usepackage{amsmath} % assumes amsmath package installed
%\usepackage{amssymb} % assumes amsmath package installed
\usepackage{CJKutf8} % for Chinese character support
\usepackage{setspace} % Chinese characters need larger line height

\title{\LARGE \bf
社群網站上之政治活動分析:以台灣\,2016\,年總統選舉為例$^{\ast}$
}

\author{\parbox{3 in}{\centering 學生:蒲郁文$^{\dagger}$}
        \hspace*{0.5 in}
        \parbox{3 in}{\centering 指導教授:孫春在\,教授$^{\ddagger}$}
        \thanks{$^{\ast}$學士論文。本研究未接收任何個人及團體贊助。}
        \thanks{$^{\dagger}$國立交通大學資訊工程學系,{\tt\small ywpu@cs.nctu.edu.tw}。}
        \thanks{$^{\ddagger}$國立交通大學資訊工程學系,{\tt\small ctsun@cs.nctu.edu.tw}。}
}

\begin{document}
\begin{CJK*}{UTF8}{bsmi}
\onehalfspacing

\maketitle
\thispagestyle{empty}
\pagestyle{empty}

%%%%%%%%%%%%%%%%%%%%%%%%%%%%%%%%%%%%%%%%%%%%%%%%%%%%%%%%%%%%%%%%%%%%%%%%%%%%%%%%

\begin{abstract}
摘要\enskip---\enskip可以輸入繁體字了。教育雲端服務平臺將進行資安設備調整作業,並於本次作業時程完成後,
進行網站功能測試,屆時網路服務將會中斷,造成您的不便,敬請見諒。獲得快速的翻譯。
\end{abstract}

\begin{keywords}
關鍵字\enskip---\enskip社群媒體、自動化內容分析、社會網絡分析、數位行動主義、政治參與
\end{keywords}

%%%%%%%%%%%%%%%%%%%%%%%%%%%%%%%%%%%%%%%%%%%%%%%%%%%%%%%%%%%%%%%%%%%%%%%%%%%%%%%%

\section*{一、緒論}

社群網站的興起不僅改變了我們的社交生活,更改變了許多國家政治與權力的運作規則。
在國內外皆有愈來愈多政治團體與公民團體透過網路來散佈其理念;
許多具引響力的社會運動也是透過網路來連結潛在支持者或動員大眾;
網路工具的靈活運用甚至還成為候選人能否贏得選舉的關鍵。

這種利用網際網路來號召群眾、表達訴求的作法被稱為數位行動主義(digital activism)。
發生在伊朗的綠色革命(Green Movement)、北非及中東的阿拉伯之春(Arab Spring)、
美國的佔領華爾街運動(Occupy Wall Street),皆被人們普遍認為與網際網路,特別是社群媒體,有密切的關連。
網際網路、社群媒體對社會與政治的影響也成為許多學者關注的新焦點。

同樣的趨勢也發生在台灣。
從洪仲丘事件、太陽花學運到反黑箱課綱運動,網際網路均對這些社會運動的促成扮演了相當關鍵的角色。
這也讓許多社會學家、政治學家、傳播學家與計算機科學家開始對台灣的數位行動主義熱潮產生興趣。
這不但是一個相當新穎的、跨領域的研究主題,這方面的研究對當今台灣社會也是相當重要的。

在台灣,推特(Twitter)較不盛行,人們使用的社群網站以臉書(Facebook)為主。
\,2016\,年總統選舉競選期間,各政黨、候選人、利益團體與意見領袖皆大量使用臉書來宣揚其政治理念。
本研究的目的便是嘗試使用資訊科技,以創新的方法分析這些使用者於臉書上的公開貼文,探討使用者於社群網站上之政治活動的模式。

\subsection*{對數位行動主義的批判}

儘管人們普遍同意數位行動主義能有效促進民主政治,這個趨勢仍然伴隨了一些潛在的隱憂。
常見對數位行動主義的批評包含:
\,(1)\,數位落差(digital divide):
由於經濟、教育等差異,有些族群缺少機會接觸資訊科技,或者沒有能力使用資訊科技。
\,(2)\,民粹主義(populism):
網路本質上便具有民粹主義、安那其(anarchy)、去中心化(decentralization)等特性,網路輿論的品質可能偏低。
\,(3)\,懶人行動主義(slacktivism):
並沒有參與什麼實際的行動,僅是在網路上分享自己關心的社會議題,便自以為自己已經對這個議題做出了貢獻。
\,(4)\,鍵擊行動主義(clicktivism):
類似懶人行動主義,若一味地追求點擊率、做網路行銷,可能反而會忽略了實際的行動。
\,(5)\,網路巴爾幹化(cyberbalkanization):
由於商業壟斷、政治審查、或是個人不願意接觸與自己立場相左的訊息等因素,使網際網路逐漸分裂、解體。

\subsection*{資訊科技與人文社會跨領域研究的挑戰與契機}

近年來,愈來愈多學者開始嘗試將計算機科學應用於人文社會科學領域,
建立了數位典藏(digital archives)、數位人文(digital humanities)、計算社會科學(computational social science)、
計算社會學(computational sociology)、社會模擬(social simulation)等新興研究領域。
然而,這樣的跨領域研究是否真的有價值一直是備受爭議的。
以下舉出一些此類研究常受到的批評,以及我從事此類研究的立場。

常見對數位人文的批評包含:
\,(1)\,只是用一些空洞的戲法來賣弄他們的研究能力,而沒有真正做出什麼新的分析。
\,(2)\,容易忽略種族、階級、性別、文化、意識型態等面向。
\,(3)\,在這個一切都已經被商品化了的現代社會中,人文學科日漸式微。但數位人文卻頂著科技的光環,而獲得了相當高的經費與聲望。
\,(4)\,研究者的背景不夠多元,其研究可能帶有某些偏見。

常見對社會模擬的批評包含:
\,(1)\,將現實社會過度地簡化為一般化的法則。
\,(2)\,模擬的成果仰賴預先建立好的模型,難以發現新的法則。
\,(3)\,預先建立的模型往往充滿偏見。
\,(4)\,模擬的成果並不能反映出現實社會的情況。

我的回應是,我認為傳統社會科學研究法仍是不可或缺的:
\,(1)\,並非所有問題都適合使用電腦來分析或模擬,有些現象無法被量化或歸納出一般化的法則。
\,(2)\,人文社會科學做的不只是資料分析、統計而已,還包含人文關懷、詮釋、思辨、批判等。
\,(3)\,因此田野工作、民族誌、訪談等研究法仍有其彌足珍貴的價值。

然而資訊科技仍然能對人文學科做出貢獻:
\,(1)\,數位人文提供了一個全新的視野來探究傳統上屬於人文社會科學領域的問題。
\,(2)\,資訊科技能幫助人文學者更有效率地獲得知識。
\,(3)\,資料探勘(data mining)、自然語言處理(natural language processing)等技術使一些過去被認為窒礙難行的研究成為可能。

\section*{二、文獻探討}

過去已有多位學者考察了社群網站在競選活動中的使用,
例如\,Larsson and Moe (2012)\,研究了\,2010\,年瑞典選舉競選期間推特的使用。\cite{c1}
該研究的作法是,蒐集競選期間含有特定\,hashtag\,的推文,進行社會網絡分析(social network analysis)。

該研究做了以下分析:
\,(1)\,不同時間推文數量的變化。
\,(2)\,各種推文類型(一般、提及、轉推)所佔的比例。
\,(3)\,列出最活躍的使用者及其推文數量,並調查其身份背景。
\,(4)\,活躍用戶的「提及」網絡圖(透過資料視覺化)。
\,(5)\,呈上,將使用者分為發送者、接受者、兩者皆是,並調查其身份背景。
\,(6)\,活躍用戶的「轉推」網絡圖(透過資料視覺化)。
\,(7)\,呈上,將使用者分為轉推者、被轉推者、兩者皆是,並調查其身份背景。

該研究發現:
\,(1)\,推文數量的變化反應了線下競選活動的情勢。
\,(2)\,大部份的推文都是由少部份的活躍用戶所發出。
\,(3)\,大部份的活躍用戶都是政治菁英,然而,平民的聲音仍具有一定的影響力。
\,(4)\,大部份的推文都是一般推文,然而,許多活躍用戶仍會在推特上與人討論。

然而該研究也具有以下限制:
\,(1)\,只有極少數的民眾有使用推特。
\,(2)\,只蒐集含有特定\,hashtag\,的推文。
\,(3)\,應分析不同情境下的使用情形,加以比較。
另外,該研究也讓我想到一個值得進一步探究的問題,那就是,如何區別真人用戶與機器用戶?

另外\,Zhang et al. (2010)\,則研究了社群網站對公民的政治態度與行為的引響。\cite{c2}
該研究參考了以下既有理論:
\,(1)\,社會資本(social capital)、大眾媒體(mass media)、人際討論(interpersonal discussion)、
社群網站(social networking sites)都會影響公民參與(civic participation)和政治參與(political participation)。
\,(2)\,影響個人的政治態度與行為的因子有:政治興趣(political interest)、政治效能(political efficacy)、
政治信任(political trust)、政黨認同(party identification)。

該研究的作法是採用電話調查,
應變數(dependent variable)為政治參與、公民參與、對政府的信任,
自變數(independent variable)為社群網站之依賴、政治之人際討論,
控制變數(control variable)為政治效能、意識形態、政治興趣、年齡、性別、種族、教育程度。
最後,進行階層式迴歸分析(hierarchical regression analysis)。

該研究發現:
\,(1)\,社群網站的使用與公民參與的提升呈現顯著的正相關,與政治參與、對政府的信任則無明顯關聯。
\,(2)\,討論政治議題有助於促進公民參與和政治參與,無助於提升對政府的信任。
\,(3)\,教育程度愈高,愈有可能參與公民活動和政治活動,但對政府的信任卻愈低。
\,(4)\,然而,政治效能感較高者,其政治信任感也較高。

然而該研究也具有以下限制:
\,(1)\,樣本可能不夠具代表性,有超過四分之三的受訪者都完全不依賴社群網站。
\,(2)\,除了公民參與/政治參與的二分法外,是否還有更適當的分類法?
另外,該研究也讓我想到兩個值得進一步探究的問題:
\,(1)\,調查結果顯示社群網站的使用無助於促進政治參與,這項發現在東方國家也成立嗎?
\,(2)\,調查結果顯示年長者較積極參與政治活動,當今台灣社會也是如此嗎?

最後\,Hosch-Dayican et al. (2016)\,則研究了\,2012\,年德國議會選舉競選期間公民如何在推特上遊說他人。\cite{c3}
其研究方法包含:

\subsection{Selecting a Template (Heading 2)}

First, confirm that you have the correct template for your paper size. This template has been tailored for output on the US-letter paper size. Please do not use it for A4 paper since the margin requirements for A4 papers may be different from Letter paper size.

\subsection{Maintaining the Integrity of the Specifications}

The template is used to format your paper and style the text. All margins, column widths, line spaces, and text fonts are prescribed; please do not alter them. You may note peculiarities. For example, the head margin in this template measures proportionately more than is customary. This measurement and others are deliberate, using specifications that anticipate your paper as one part of the entire proceedings, and not as an independent document. Please do not revise any of the current designations

\section{MATH}

Before you begin to format your paper, first write and save the content as a separate text file. Keep your text and graphic files separate until after the text has been formatted and styled. Do not use hard tabs, and limit use of hard returns to only one return at the end of a paragraph. Do not add any kind of pagination anywhere in the paper. Do not number text heads-the template will do that for you.

Finally, complete content and organizational editing before formatting. Please take note of the following items when proofreading spelling and grammar:

\subsection{Abbreviations and Acronyms} Define abbreviations and acronyms the first time they are used in the text, even after they have been defined in the abstract. Abbreviations such as IEEE, SI, MKS, CGS, sc, dc, and rms do not have to be defined. Do not use abbreviations in the title or heads unless they are unavoidable.

\subsection{Units}

\begin{itemize}

\item Use either SI (MKS) or CGS as primary units. (SI units are encouraged.) English units may be used as secondary units (in parentheses). An exception would be the use of English units as identifiers in trade, such as 3.5-inch disk drive.
\item Avoid combining SI and CGS units, such as current in amperes and magnetic field in oersteds. This often leads to confusion because equations do not balance dimensionally. If you must use mixed units, clearly state the units for each quantity that you use in an equation.
\item Do not mix complete spellings and abbreviations of units: Wb/m2 or weber per square meter, not webers/m2.  Spell out units when they appear in text: . . . a few henries, not . . . a few H.
\item Use a zero before decimal points: 0.25, not .25. Use cm3, not cc. (bullet list)

\end{itemize}


\subsection{Equations}

The equations are an exception to the prescribed specifications of this template. You will need to determine whether or not your equation should be typed using either the Times New Roman or the Symbol font (please no other font). To create multileveled equations, it may be necessary to treat the equation as a graphic and insert it into the text after your paper is styled. Number equations consecutively. Equation numbers, within parentheses, are to position flush right, as in (1), using a right tab stop. To make your equations more compact, you may use the solidus ( / ), the exp function, or appropriate exponents. Italicize Roman symbols for quantities and variables, but not Greek symbols. Use a long dash rather than a hyphen for a minus sign. Punctuate equations with commas or periods when they are part of a sentence, as in

$$
\alpha + \beta = \chi \eqno{(1)}
$$

Note that the equation is centered using a center tab stop. Be sure that the symbols in your equation have been defined before or immediately following the equation. Use (1), not Eq. (1) or equation (1), except at the beginning of a sentence: Equation (1) is . . .

\subsection{Some Common Mistakes}
\begin{itemize}


\item The word data is plural, not singular.
\item The subscript for the permeability of vacuum ?0, and other common scientific constants, is zero with subscript formatting, not a lowercase letter o.
\item In American English, commas, semi-/colons, periods, question and exclamation marks are located within quotation marks only when a complete thought or name is cited, such as a title or full quotation. When quotation marks are used, instead of a bold or italic typeface, to highlight a word or phrase, punctuation should appear outside of the quotation marks. A parenthetical phrase or statement at the end of a sentence is punctuated outside of the closing parenthesis (like this). (A parenthetical sentence is punctuated within the parentheses.)
\item A graph within a graph is an inset, not an insert. The word alternatively is preferred to the word alternately (unless you really mean something that alternates).
\item Do not use the word essentially to mean approximately or effectively.
\item In your paper title, if the words that uses can accurately replace the word using, capitalize the u; if not, keep using lower-cased.
\item Be aware of the different meanings of the homophones affect and effect, complement and compliment, discreet and discrete, principal and principle.
\item Do not confuse imply and infer.
\item The prefix non is not a word; it should be joined to the word it modifies, usually without a hyphen.
\item There is no period after the et in the Latin abbreviation et al..
\item The abbreviation i.e. means that is, and the abbreviation e.g. means for example.

\end{itemize}


\section{USING THE TEMPLATE}

Use this sample document as your LaTeX source file to create your document. Save this file as {\bf root.tex}. You have to make sure to use the cls file that came with this distribution. If you use a different style file, you cannot expect to get required margins. Note also that when you are creating your out PDF file, the source file is only part of the equation. {\it Your \TeX\ $\rightarrow$ PDF filter determines the output file size. Even if you make all the specifications to output a letter file in the source - if you filter is set to produce A4, you will only get A4 output. }

It is impossible to account for all possible situation, one would encounter using \TeX. If you are using multiple \TeX\ files you must make sure that the ``MAIN`` source file is called root.tex - this is particularly important if your conference is using PaperPlaza's built in \TeX\ to PDF conversion tool.

\subsection{Headings, etc}

Text heads organize the topics on a relational, hierarchical basis. For example, the paper title is the primary text head because all subsequent material relates and elaborates on this one topic. If there are two or more sub-topics, the next level head (uppercase Roman numerals) should be used and, conversely, if there are not at least two sub-topics, then no subheads should be introduced. Styles named Heading 1, Heading 2, Heading 3, and Heading 4 are prescribed.

\subsection{Figures and Tables}

Positioning Figures and Tables: Place figures and tables at the top and bottom of columns. Avoid placing them in the middle of columns. Large figures and tables may span across both columns. Figure captions should be below the figures; table heads should appear above the tables. Insert figures and tables after they are cited in the text. Use the abbreviation Fig. 1, even at the beginning of a sentence.

\begin{table}[h]
\caption{An Example of a Table}
\label{table_example}
\begin{center}
\begin{tabular}{|c||c|}
\hline
One & Two\\
\hline
Three & Four\\
\hline
\end{tabular}
\end{center}
\end{table}


   \begin{figure}[thpb]
      \centering
      \framebox{\parbox{3in}{We suggest that you use a text box to insert a graphic (which is ideally a 300 dpi TIFF or EPS file, with all fonts embedded) because, in an document, this method is somewhat more stable than directly inserting a picture.
}}
      %\includegraphics[scale=1.0]{figurefile}
      \caption{Inductance of oscillation winding on amorphous
       magnetic core versus DC bias magnetic field}
      \label{figurelabel}
   \end{figure}
   

Figure Labels: Use 8 point Times New Roman for Figure labels. Use words rather than symbols or abbreviations when writing Figure axis labels to avoid confusing the reader. As an example, write the quantity Magnetization, or Magnetization, M, not just M. If including units in the label, present them within parentheses. Do not label axes only with units. In the example, write Magnetization (A/m) or Magnetization {A[m(1)]}, not just A/m. Do not label axes with a ratio of quantities and units. For example, write Temperature (K), not Temperature/K.

\section{CONCLUSIONS}

A conclusion section is not required. Although a conclusion may review the main points of the paper, do not replicate the abstract as the conclusion. A conclusion might elaborate on the importance of the work or suggest applications and extensions. 

\addtolength{\textheight}{-12cm}  % This command serves to balance the column lengths
                                  % on the last page of the document manually. It shortens
                                  % the textheight of the last page by a suitable amount.
                                  % This command does not take effect until the next page
                                  % so it should come on the page before the last. Make
                                  % sure that you do not shorten the textheight too much.

%%%%%%%%%%%%%%%%%%%%%%%%%%%%%%%%%%%%%%%%%%%%%%%%%%%%%%%%%%%%%%%%%%%%%%%%%%%%%%%%



%%%%%%%%%%%%%%%%%%%%%%%%%%%%%%%%%%%%%%%%%%%%%%%%%%%%%%%%%%%%%%%%%%%%%%%%%%%%%%%%



%%%%%%%%%%%%%%%%%%%%%%%%%%%%%%%%%%%%%%%%%%%%%%%%%%%%%%%%%%%%%%%%%%%%%%%%%%%%%%%%
\section*{APPENDIX}

Appendixes should appear before the acknowledgment.

\section*{ACKNOWLEDGMENT}

The preferred spelling of the word acknowledgment in America is without an e after the g. Avoid the stilted expression, One of us (R. B. G.) thanks . . .  Instead, try R. B. G. thanks. Put sponsor acknowledgments in the unnumbered footnote on the first page.

%%%%%%%%%%%%%%%%%%%%%%%%%%%%%%%%%%%%%%%%%%%%%%%%%%%%%%%%%%%%%%%%%%%%%%%%%%%%%%%%

\bibliographystyle{ieeetr}
\bibliography{citation}

\end{CJK*}
\end{document}
